\documentclass[12pt]{article}
\usepackage[margin=1.0cm]{geometry}

\title{\textsc{Seminario de Astronom\'ia,\\ Planetario de Bogot\'a.}}

\begin{document}
\date{}
\maketitle

\section*{\textsc{Objetivo:}}
Crear un espacio de socializaci\'on, integraci\'on y difusi\'on de la investiagaci\'on astronom\'ica en el pa\'is
a cargo de los estudiantes.     

\section*{\textsc{Metodolog\'ia:}}
El seminario se llevar\'a acabo cada ocho d\'ias en el planetario distrital de Bogot\'a.
Durante cada sesi\'on se presentar\'an 2  trabajos de estudiantes con una duraci\'on de 40 min mas 15 min de 
preguntas cada uno. Adicionalmente pueden haber presentaciones de art\'iculos cient\'ificos y de proyectos de divulgaci\'on.


\section*{\textsc{Programaci\'on:}}

\begin{table}[hp]
\begin{center}
\begin{tabular}{c c c c}
\textsc{Expositor} & \textsc{T\'itulo} & \textsc{Fecha} & \textsc{Instituci\'on} \\
\hline
\hline
\\
Juan Sebastian Castellanos & "Fulguraciones\ Solares" & Abril 5 & UNAL \\
Cesar Daniel Peralta & "Teorias f(R) en Cosmolog\'ia" & Abril 5 & UNAL \\
Diego Uma\~na & "Fotografiando\ Supernova\ M82" & Abril 12& UniAndes \\
Christian Poveda & Por Confirmar & Abril 12 & UniAndes\\
Andres Felipe Ramos & "Distribuci\'on Espacial de los Hidrocarburos  & Abril 26& UNAL \\
 & Aromaticos Policiclicos en CED201" & & \\
Christian D. Rodriguez & Por Confirmar & Abril 26& UNAL \\
Maria Fernanda Gomez & "Formaci\'on de Agujeros Negros & Mayo 3 & UniAndes \\
& Supermasivos" & & \\
Alejandro Cardenas & "Agujeros Negros & Mayo 3 & UNAL \\
& (Radiaci\'on Hawking)" & & \\
Wilmar Fajardo & Por Confirmar & Mayo 10 & UNAL\\
Juan Nicol\'as Garavito C. & "Galaxias a Alto Redshift (LAEs)"& Mayo 10 & UniAndes \\
Felipe Gomez & Por Confirmar & Junio & UniAndes \\
Nicol\'as Amado P. & Por Confirmar & Por Confirmar & U. Distrital\\ 
Seditsira Quintero & Por Confirmar &  Por Confirmar & UNAL \\
\hline
\end{tabular}
\end{center}
\end{table}

\section*{Otras Actividades:}
\begin{itemize}
\item Presentaci\'on Posters a cargo de los estudiantes de UniAndes, Proyectos $AstroLunch$ \\
(\verb"https://github.com/forero/AstroSeminarUniandes").
\end{itemize}

\end{document}